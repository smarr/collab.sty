\documentclass{article}
\usepackage{collab}

\collabAuthor{sm}{red}{Stefan Marr}
\collabAuthor{op}{Magenta}{Other Person}


\begin{document}

\section{Editing Commands}

\begin{itemize}
  \item \verb|\spelling{wrongly spelled}| \spelling{wrongly spelled}

  \item \verb|\rephrase{please rephrase}| \rephrase{please rephrase}

  \item \verb|\ugh{please rephrase}| \ugh{please rephrase}

  \item \verb|\ins{this should be edit}| \ins{this should be edit}

  \item \verb|\del{suggestion to remove}| \del{suggestion to remove}

  \item \verb|\chg{old text}{new text}| \chg{old text}{new text}
  
  \item \verb|\wip{work in progress}| \wip{work in progress}
\end{itemize}

\section{General Todo Commands}

\begin{itemize}
  \item \verb|\todo{needs to be done}|  \todo{needs to be done}
  \item \verb|\todoref{add reference}| \todoref{add reference}
  \item \verb|\DONE{done todo}| \DONE{done todo}
  \item \verb|\REM{should remember}| \REM{should remember}
  \item \verb|\rev{2}{this is stupid}| \rev{2}{this is stupid}
  \item \verb|\fix{Fix this}| \fix{Fix this}
\end{itemize}

For elaborated notes:

\begin{verbatim}
\begin{note}
- abcde
- more notes
\end{note}
\end{verbatim}

\begin{note}
- abcde
- more notes
\end{note}



\section{Commands per Person}

The \verb|collabAuthor| macro can be given a shorthand, which is used to create macros:
\verb|\collabAuthor{sm}{red}{Stefan Marr}|

\begin{itemize}
  \item \verb|\sm{general comment}| \sm{general comment}
  \item \verb|\smtodo{please do}| \smtodo{please do}
  \item \verb|\smdone{please do}| \smdone{please do}
  \item \verb|\smnote{hmmm}| \smnote{hmmm}
  \item \verb|\smQ{What's this}| \smQ{What's this}
\end{itemize}

Or perhaps \verb|\collabAuthor{op}{purple}{Other Person}|:

\begin{itemize}
  \item \verb|\op{general comment}| \op{general comment}
  \item \verb|\optodo{please do}| \optodo{please do}
  \item \verb|\opdone{please do}| \opdone{please do}
  \item \verb|\opnote{hmmm}| \opnote{hmmm}
  \item \verb|\opQ{What's this}| \opQ{What's this}
\end{itemize}

\end{document}
